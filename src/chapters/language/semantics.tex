\section{Semantics}\label{sec:semantics}

In order to talk about program properties in our language, we first
need to define its \emph{semantics}. In the following section we
introduce both a collecting semantics in order to reason about program
\emph{invariants} and a small step semantics, in order to reason about
program \emph{execution}

\begin{definition}[Semantics of Basic Expressions]
  Let \emph{environments} be the maps from the set of variables to
  their numerical value:
  \(\env \defin \{\rho \mid \rho : \Var \to \n\}\). For basic
  expressions \(e \in \expr\) the \emph{concrete semantics}
  \(\bsem{ \cdot} : \expr \to \env \to \smashed{\env}\) is
  inductively defined by:
  % align* to horizontally align multiple equations
  \begin{align*}
    \bsem{\var\in S} \rho & \defin \begin{cases} \rho & \rho(\var)\in S \\ \bot & \text{otherwise} \end{cases} \\
    \bsem{\var:= k} \rho & \defin \rho [\var\mapsto k] \\
    \bsem{\var:= \var[y] + k} \rho & \defin \begin{cases} \rho [\var\mapsto \rho(\var[y]) + k] & \rho \neq \bot \\ \bot & \text{otherwise}\end{cases} \\
    \bsem{\var:= \var[y] - k} \rho & \defin \begin{cases} \rho [\var\mapsto \rho(\var[y]) - k] & \rho \neq \bot \\ \bot & \text{otherwise}\end{cases} \\
  \end{align*}
  where with \(\smashed{\env}\) we mean \(\env \cup \{\bot\}\).
\end{definition}

The next building block is the concrete collecting semantics for the
language, it associates each program in \(\imp\) to a function which,
given a set of initial environments \(X\) ``collects'' the set of
final states produced by executing the program from \(X\).

\begin{definition}[Concrete collecting semantics]\label{de:concretesem}
  Let \(\dom \defin \langle \poset{\env} , \subseteq \rangle\) be a
  complete lattice called \emph{concrete collecting domain}. The
  \emph{concrete collecting semantics} for \(\imp\) is given by the
  total function \(\sem{\cdot} : \imp \to \dom \to \dom\) which maps
  each program \(\com \in \imp\) to a total function over the complete
  lattice \(\dom\), inductively defined as follows: given
  \(X \in \dom\)

  \begin{align*}
    \sem{\com[e]} X & \defin \{\bsem{\com[e]} \rho \mid \rho \in X,
                      \bsem{\com[e]} \rho \neq \bot\} \\
    \sem{\com[C_1] + \com[C_2]} X & \defin \sem{\com[C_1]} X \cup \sem{\com[C_2]} X \\
    \sem{\com[C_1] ; \com[C_2]} X & \defin \sem{\com[C_2]}(\sem{\com[C_1]} X) \\
    \sem{\com[C^*]} X & \defin \bigcup_{i \in \n} \sem{\com[C]}^i X \\
    \sem{\fix{C}} X & \defin \lfp(\lambda Y \in\poset{\env} . (X \cup \sem{\com}Y))
  \end{align*}
\end{definition}

We observe that the semantics we described is additive:

\begin{observation}[Semantics Additivity]
  Given \(\com\in\imp\), \(X,Y \in \dom\),
  \[\sem{\com}(X\cup Y) = \sem{\com}X \cup \sem{\com}Y\]
\end{observation}

\begin{proof}
  \dots
\end{proof}

% Subsections in this case are directly in the section file, but you
% can also make a folder for the section and \subimport the
% subsections, it's up to you

\subsection{Syntactic sugar}\label{sub:sugar}
We define some syntactic sugar for the language. In the next chapters
we will often use the syntactic sugar instead of its real equivalent
for the sake of simplicity.
% 
% The Todo package allows for annotations directly rendered in the pdf
\todo[inline]{Add more syntactic sugar for the examples}
%
\begin{align*}
  \var \in \interval{a}{b} & = \var \in S & \text{with } S = \interval{a}{b}, \text{ decidable}\\
  \var \leq k & = \var \in (-\infty, k]\\
  \var > k & = \var \in [k+1, + \infty) \\
  \tru & = \var \in \n \\
  \ff & = \var \in \emptyset \\
  \var \in S_1 \vee \var \in S_2 & = (\var \in S_1) + (\var \in S_2) \\ 
  \var \in S_1 \wedge \var \in S_2 & = (\var \in S_1);(\var \in S_2) \\
  \var \not\in S & = \var \in \neg S \\
  \ifte{b}{C_1}{C_2} & = (\com[e];\com_1) + (\com[\neg e];\com_2) \\
  \while{b}{C} & = \fix{\com[e];\com};\neg\com[e] \\
  \pplus{\var} & = \var := \var + 1
\end{align*}

\subsection{Small step semantics}\label{sub:sos}

Now that we have defined the collecting semantics to express program
properties, we need the small step semantics to talk about program
execution. We start by defining \emph{program states}:
\(\state \defin \imp \times \env\) tuples of programs and program
environments.  With states we can define our small step semantics:

% \infer for logical inferences and sequent calculus

\begin{definition}[Small step semantics]\label{def:sosem}
  The small step transition relation for the language \(\imp\)
  \(\to : \state \times (\state \cup \env)\) is defined by the
  following rules:
  \begin{equation*}
    \infer[\text{expr}]
          {\stt{\com[e], \rho} \to \bsem{e}\rho}
          {\bsem{e}\rho \neq \bot}
  \end{equation*}
  
  \begin{equation*}
    \infer[\text{sum}_1]
    {\stt{\com[C_1 + C_2], \rho} \to \stt{\com[C_1], \rho}}
    {} \quad
    \infer[\text{sum}_2]
          {\stt{\com[C_1 + C_2], \rho} \to \stt{\com[C_2], \rho}}
          {}
  \end{equation*}
  
  \begin{equation*}
    \infer[\text{comp}_1]
          {\stt{\com[C_1; C_2], \rho} \to \stt{\com[C_1'; C_2], \rho'}}
          {\stt{\com[C_1], \rho} \to \stt{\com[C_1'], \rho'}} \quad
    \infer[\text{comp}_2]
          {\stt{\com[C_1; C_2], \rho} \to \stt{\com[C_2], \rho'}}
          {\stt{\com[C_1], \rho} \to \rho' }
  \end{equation*}

  \begin{equation*}
    \infer[\text{star}]
          {\stt{\com[C^*], \rho} \to \stt{\com[C;C^*], \rho}}
          {} \quad
    \infer[\text{star}_{\text{fix}}]
          {\stt{\com[C^*], \rho} \to \rho}
          {}
  \end{equation*}
\end{definition}

\noindent
In the following chapters we will usually use the following notation
to talk about program execution:

\begin{itemize}
\item \(\to^+\) is the transitive closure of the relation \(\to\);
\item \(\to^*\) is the reflexive and transitive closure of the
  relation \(\to\).
\end{itemize}
\noindent

With the following lemma we introduce a link between the small step
semantics and the concrete collecting semantics: the invariant of a
program is the collection of all the environments the program halts on
when executing.

\begin{lemma}\label{le:link}
  For any \(\com\in\imp, X \in \poset{\env}\)
  \begin{equation*}
    \sem{\com}X = \{\rho' \in \env \mid \rho \in X, \stt{\com, \rho} \to^* \rho'\}
  \end{equation*}
\end{lemma}

where \(\to^*\) is the reflexive and transitive closure of the \(\to\)
relation.

\begin{proof}
  \dots
\end{proof}
